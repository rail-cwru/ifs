\chapter { Conclusion }

In this thesis, we proposed and implemented a collaborative framework for feature selection. By designing visualizations and interactions that facilitate the communication between the user and the machine learning algorithm, we are able to integrate the user's prior knowledge and the user into the feature selection process.

Our feature selection system is a collaborative process that enables the user to encode their prior knowledge. Through interactions with the visualization, the user can visually encode the importance of a feature to the target variable as well as the ranking of the feature according to their importance. The first step of the system is to express and establish prior knowledge and so the system can incorporate additional steps, visuals that represent the user's prior knowledge.

Our implemented feature selection framework integrates causal discovery into the feature selection process and enables the user to integrate their prior knowledge about the causal relationship between features. A causal discovery algorithm incorporates user-provided information about feature importance to build a causal network. Then the user adds, removes, and/or reverse edges in the network to reflect what they previously know about causal relationships in the dataset. Through collaboration facilitated by visualization, the user and system create a causal network that may be more accurate of the underlying data generation mechanisms. The causal network can then help the system and user filter for predictive features.
Moreover, our feature selection step enables the users to dynamically explore the feature space while accessing how consistent the current feature set is to previously express information. By dragging the visual elements representing features, users can quickly build a feature set and efficiently make modifications. The examples are also represented as visual elements rather than described by their feature vector to leverage the user's visual processing power and help them efficiently identify patterns in the dataset. Moreover, a feature analysis page is provided to incorporate the user's prior information. Metrics such as Markov blanket consistency score and rank loss communicates to the user how consistent the current feature set is to their prior knowledge and whether it would possibly be a predictive feature set.

Lastly, our system is designed for the iterative process of classification. Machine learning practitioners have to create a classifier, assess its current performance, make modifications to its inputs such as the feature selection, and then start the process again. We incorporate a performance analysis step that incorporates common and visually effective graphs such as a gradient color coded confusion matrix and the ROC Curve graph. The user can create multiple classifiers from different feature selections. The system keeps track of the previous feature selections and the performances of previous classifiers so that the user can make quick changes to previous feature selections as well as compare the performances of previous classifiers.

After the system was implemented, we designed an evaluation study to access its effectiveness at facilitating collaborative feature selection, interpretability of the visualizations, and its ability to facilitate exploration of feature space. A separate system, version B, that excluded user expressed prior knowledge was created to evaluate the effects of integrating prior knowledge in the feature selection process. A series of tasks were designed to evaluate each part of the process; 15 participants evaluated the full system, while 10 participants evaluated version B.

Our study showed that our system allowed for efficient and effective integration of prior knowledge. Participants communicated feature importance to the system and collaborated with the system to build a possible causal network. Moreover, participants were able to visually process patterns in the datasets; however, some expressed difficulties in identifying patterns. Participants were able to effectively and efficiently explore the feature space; they created many feature selections and classifiers before identifying a set of features for the classification task.

By incorporating and establishing prior knowledge, the participants were able to focus on a subset of the feature space. Although participants using version B and those using the complete system achieved similar performing classifiers, their behavior during the feature selection step differed. Participants using the complete system averaged more time for each classifier because they were utilizing the feature analysis metrics to assess their current feature selection before creating the classifier. Moreover, they started with a default set of features consisting of the features that directly influence the target variable in the causal graph, which made them focus on certain feature sets and make modifications to the default feature set. For more complex datasets, the feature analysis metrics and causal features will guide the user towards a subspace of predictive feature sets.

From the evaluation study, we also observed how the system and our design can be improved. First, two participants commented that the feature selection visual can be difficult to interpret because the orientation of the feature among the other features can either reveal or hide patterns between the features. We hope to investigate and evaluate other methods of displaying the flow of examples from one feature to another; we know that when condensing a lot of information into a visualization some information will be lost or deformed and we seek to minimize that. Second, the color of the examples affect the feature selection visuals, we should use more neutral colors or use colors that relate to the target label. Moreover, we can incorporate other types of prior knowledge.

We can use a different causal discovery algorithm to evaluate whether the causal discovery algorithm makes a difference to the feature selection. Since we observed that prior knowledge is useful for orienting edges in causal networks, we can also research and design a more comprehensive framework for interactive causal discovery. Furthermore, we need to incorporate other metrics such as those from filter-based feature selection methods to help user filter for predictive features. We need a metric that communicates the amount of redundant information between pairs of features which can help the user identify redundant information in the feature set.

To sum up, feature selection can greatly benefit from collaboration with humans specifically for integrating prior knowledge.  The proposed feature selection framework integrates effective and efficient visualizations and interactions for establishing prior knowledge and exploring the space of features. With more useful metrics and improvements in the design of visual elements, practitioners can help machine learning algorithms identify a feature set for their classification problem.  The application of interactive machine learning, visual analytics, and novel feature selection techniques have great research potential.
